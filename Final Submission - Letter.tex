\documentclass[12pt,stdletter,orderfromtodate,sigleft]{newlfm}
\usepackage{blindtext, xfrac}
 
\newlfmP{dateskipbefore=20pt}
\newlfmP{sigsize=20pt}
\newlfmP{sigskipbefore=20pt}
 
\newlfmP{Headlinewd=0pt,Footlinewd=0pt}
 
\addrfrom{%
    Mathematical Contest in Modeling\\
    COMAP
}
 
\addrto{%
    Chief Administrator\\
    DEA/NFLI Database
}
 
\dateset{January 28, 2019}
 
\greetto{To Whom It May Concern,}

\closeline{Sincerely, \\
\bigskip
Team \#1926726}
 
\begin{document}
\begin{newlfm}

After analyzing the data on opioid use in Ohio, Kentucky, West Virginia, Virginia, and Pennsylvania, our team has noted several important trends. 

First, we hypothesize the opioid epidemic originated in the following counties: Jefferson, Kentucky; Cuyahoga, Ohio; Pittsburgh, Pennsylvania; Richmond, Virginia; and Kanawha, West Virginia. We noted that the number of reported drug incidents tends to increase with the size of the population. However, there are also a variety of unknown variables causing spikes and decreases in the data that require further analysis.

Second, we noted that there is no correlation between education and drug use. Therefore, increasing ad campaigns and drug education will likely have little to no effect on the number of people who are users. 

Our proposed solution to counter the opioid crisis is to increase the availability and effectiveness of government rehab centers. We could improve rehabilitation centers by having stricter regulations and guidelines. Increasing rehab availability will encourage more people to seek treatment, decreasing the length of addiction. Increasing the effectiveness of the rehabilitation centers will decrease the possibility that individuals will relapse. 

We urge you take our findings into consideration when implementing new tactics to battle the opioid crisis.  

\end{newlfm}
\end{document}